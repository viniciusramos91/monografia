The need for greater processing power and storage, caused by the complexity of today's applications and systems, has given space to the development of new paradigms in computing. Thus, it created the concept of Cloud Computing. This new way of providing computing services has enabled the development and creation of various applications that share different technologies and service providers. In this scenario, applications in Bioinformatics has benefited from this new platform due to demand increasing amounts of processing. The BioNimbuZ, a Bioinformatics workflow execution platform developed at the University of Brasilia by the student Hugo Saldanha uses the Cloud Computing paradigm to process application flows in different computer services providers, such as Microsoft Azure and Amazon EC2. Thus, it is necessary to manage the execution of these flows (worflows) since its submission to the system until their completion, such as provide an interface for the user to have access to these services. This paper proposes improvements in the management and control of workflows undergoing BioNimbuZ platform implementing a Scientific Workflow Management System based on web technologies.