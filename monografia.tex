%%%%%%%%%%%%%%%%%%%%%%%%%%%%%%%%%%%%%%%%
% 		  Classe do documento          %
%%%%%%%%%%%%%%%%%%%%%%%%%%%%%%%%%%%%%%%%
\documentclass[engenharia]{UnB-CIC}%

%%%%%%%%%%%%%%%%%%%%%%%%%%%%%%%%%%%%%%%%
%			   Includes				   %
%%%%%%%%%%%%%%%%%%%%%%%%%%%%%%%%%%%%%%%%
\usepackage{pdfpages} % incluir PDFs, usado no apêndice
\usepackage{float}
\usepackage[brazil,american]{babel}
\usepackage{booktabs}
\usepackage{graphicx}
\usepackage{array}
\usepackage{multirow}

%%%%%%%%%%%%%%%%%%%%%%%%%%%%%%%%%%%%%%%%
%           Cores dos links   	   	   %
%%%%%%%%%%%%%%%%%%%%%%%%%%%%%%%%%%%%%%%%

% Veja o arquivos cores.tex se quiser ver que outras cores estão
% pré-definidas.  Utilizando o comando \hypersetup abaixo nós
% evitamos aquelas caixas vermelhas feias em volta dos links.

%%%%%%%%%%%%%%%%%%%%%%%%%%%%%%%%%%%%%%%%
% Cores do estilo Tango
%%%%%%%%%%%%%%%%%%%%%%%%%%%%%%%%%%%%%%%%

\definecolor{LightButter}{rgb}{0.98,0.91,0.31}
\definecolor{LightOrange}{rgb}{0.98,0.68,0.24}
\definecolor{LightChocolate}{rgb}{0.91,0.72,0.43}
\definecolor{LightChameleon}{rgb}{0.54,0.88,0.20}
\definecolor{LightSkyBlue}{rgb}{0.45,0.62,0.81}
\definecolor{LightPlum}{rgb}{0.68,0.50,0.66}
\definecolor{LightScarletRed}{rgb}{0.93,0.16,0.16}
\definecolor{Butter}{rgb}{0.93,0.86,0.25}
\definecolor{Orange}{rgb}{0.96,0.47,0.00}
\definecolor{Chocolate}{rgb}{0.75,0.49,0.07}
\definecolor{Chameleon}{rgb}{0.45,0.82,0.09}
\definecolor{SkyBlue}{rgb}{0.20,0.39,0.64}
\definecolor{Plum}{rgb}{0.46,0.31,0.48}
\definecolor{ScarletRed}{rgb}{0.80,0.00,0.00}
\definecolor{DarkButter}{rgb}{0.77,0.62,0.00}
\definecolor{DarkOrange}{rgb}{0.80,0.36,0.00}
\definecolor{DarkChocolate}{rgb}{0.56,0.35,0.01}
\definecolor{DarkChameleon}{rgb}{0.30,0.60,0.02}
\definecolor{DarkSkyBlue}{rgb}{0.12,0.29,0.53}
\definecolor{DarkPlum}{rgb}{0.36,0.21,0.40}
\definecolor{DarkScarletRed}{rgb}{0.64,0.00,0.00}
\definecolor{Aluminium1}{rgb}{0.93,0.93,0.92}
\definecolor{Aluminium2}{rgb}{0.82,0.84,0.81}
\definecolor{Aluminium3}{rgb}{0.73,0.74,0.71}
\definecolor{Aluminium4}{rgb}{0.53,0.54,0.52}
\definecolor{Aluminium5}{rgb}{0.33,0.34,0.32}
\definecolor{Aluminium6}{rgb}{0.18,0.20,0.21}

\hypersetup{
  colorlinks=true,
  linkcolor=SkyBlue,
  citecolor=SkyBlue,
  filecolor=SkyBlue,
  urlcolor= SkyBlue
}

%%%%%%%%%%%%%%%%%%%%%%%%%%%%%%%%%%%%%%%%
% 	    Informações do Trabalho		   %
%%%%%%%%%%%%%%%%%%%%%%%%%%%%%%%%%%%%%%%%
\orientador[a]{\prof[a] \dr[a] Aletéia Patrício Favacho de Araújo}{CIC/UnB}%

\coordenador{\prof \dr Andre Drummond}{CIC/UnB}%
\diamesano{12}{fevereiro}{2016}%

\membrobanca{\prof \drª Maria Emília Machado Telles Walter}{CIC/UnB}
\membrobanca{\prof \drª Genaina Nunes Rodrigues}{CIC/UnB}

\autor{Vinicius de A.}{Ramos}%

\titulo{Um Sistema Gerenciador de Workflows Científicos Para a Plataforma de Nuvens Federadas BioNimbuZ}%

\palavraschave{Computação em Nuvem, \textit{Workflow} Científico, Sistema Gerenciador de \textit{Workflows} Científicos, Tecnologias \textit{web}}%
\keywords{Cloud Computing, Scientific Workflow, Scientific Workflow Management System, web technologies}%

\CDU{004.4}%

\newcommand{\unbcic}{\texttt{UnB-CIC}}%
\newcommand{\code}[1]{{\color[HTML]{7e0854} \texttt{#1}}}

%%%%%%%%%%%%%%%%%%%%%%%%%%%%%%%%%%%%%%%%
% 				 Texto				   %
%%%%%%%%%%%%%%%%%%%%%%%%%%%%%%%%%%%%%%%%
\begin{document}%
    \capitulo{capitulo1}{Introdução} 
    \capitulo{capitulo2}{Computação em Nuvem}%
    \capitulo{capitulo3}{Workflow Científico}%
    \capitulo{capitulo4}{BioNimbuZ}%
    \capitulo{capitulo5}{Sistema Gerenciador de Workflows Científicos para o BioNimbuZ}
    \capitulo{capitulo6}{Resultados Obtidos}
    \capitulo{capitulo7}{Conclusões e Trabalhos Futuros}
    
%%%%%%%%%%%%%%%%%%%%
% Medida Paleativa %
%%%%%%%%%%%%%%%%%%%%
\newpage

\begin{center}
   	\textbf{\large{Referência Bibliográfica}} \\
    \textit{Temporário}
\end{center}  

\begin{itemize}
    \item [1] Cloud Computing and Grid Computing 360-Degree Compared
	\item [2] A Taxonomy and Survey of Cloud Computing Systems
	\item [3] BioNimbus: uma arquitetura de federação de nuvens computacionais híbrida para a execução de workflows de Bioinformática
	\item [4] Scientific Process Automation and Workflow Management
	\item [5] A Break In The Clouds: Towards a Cloud Definition
	\item [6] Distributed Systems and Recent Innovations: Challenges and Benefits
	\item [9] Consistent Global States of Distributed Systems: Fundamental Concepts and Mechanisms
	\item [10] Time, Clocks and the Ordering of Events in a Distributed Systems.
	\item [11] Cloud Computing vs. Grid Computing 
	\item [12] The Anatomy of The Grid
	\item [13] Cluster Computing - High Performance, High-Availability and High Throuhput Processing on a Network of Computers
	\item [14] Cluster Computing: Architectures, Operating Systems, Parallel Processing and Programming Languages
	\item [15] Cloud Computing - State of the Art and Research Challenges
	\item [16] The NIST Definition of Cloud Computing
    \item [17] Xen Project, http://www.xenproject.org/ (acessado em 11/01/2016)
    \item [18] KVM, http://www.linux-kvm.org/page/Main\_Page (acessado em 11/01/2016)
    \item [19] VMWare, http://www.vmware.com/br (acessado em 11/01/2016)
    \item [20] How to Enhance Cloud Architectures to Enable Cross-Federation
    \item [21] Provenance Collection Support in the Kepler Scientific Workflow System
    \item [22] http://www.csc.ncsu.edu/faculty/mpsingh/papers/databases/workflows/sciworkflows.html (acessado em 12/01/2016)	
    \item [23] Towards a Model of Provenance and User Views in Scientific Workflows	
    \item [24] Scientific Workflow Management and the KEPLER System	
    \item [25] WOODSS — a spatial decision support system based on workflows	
    \item [26] E. White, L. McMillan, P. Romanski, M. O'Gara, and J. Bloomberg. Inter-cloud peering points. http://cloudcomputing.sys-con.com/node/1658700, 2010. vii, 13
    \item [27] Dropbox http://www.dropbox.com/
    \item [28] Amazon EC2 https://aws.amazon.com/pt/ec2/
    \item [29] Heroku https://www.heroku.com/
    \item [30] Scientific Process Automation and Workflow Management
    \item [31] Scientific Workflows - Business as Usual?
    \item [32] JSF https://javaserverfaces.java.net/
    \item [33] Primefaces http://www.primefaces.org/
    \item [34] M. Ghanem, V. Curcin, P. Wendel, and Y. Guo, “Building and using analytical workflows in discovery net,” in Data mining on the Grid, W. Dubitzky, Ed. John Wiley and Sons, 2008.
    \item [35] D. Hull, K. Wolstencroft, R. Stevens, C. Goble, M. R. Pocock, P. Li,and T. Oinn, “Taverna: A tool for building and running workflows of services,” Nucleic Acids Research, vol. 34, pp. W729–W732, 2006, web Server Issue.
	\item [36] I. Taylor, M. Shields, I. Wang, and A. Harrison, “Visual Grid Workflow in Triana”, Journal of Grid Computing, vol. 3, no. 3-4, pp. 153–169, September 2005. [Online]. Available:
\begin{verbatim}
\http://www.springerlink.com/openurl.asp?genre=article&issn=1570- 
7873&volume=3&issue=3&spage=153
\end{verbatim}
    \item [37] B. Ludascher, I. Altintas, C. Berkley, D. Higgins, E. Jaeger, M. Jones, E. A. Lee, J. Tao, and Y. Zhao, “Scientific workflow management and the kepler system: Research articles,” Concurr. Comput. : Pract. Exper., vol. 18, no. 10, pp. 1039–1065, 2006.
    \item [38] Pegasus Mapping Scientific Workflows onto the Grid 
	\item [39] The Pegasus Portal Web Based Grid Computing
    \item [40] Use a Cabeça! Padrões de Projeto
    \item [41] Spring MVC https://spring.io (acessado em 18/01)
    \item [42] Struts https://struts.apache.org (acessado em 18/01)
    \item [43] Play https://www.playframework.com
	\item [44] IDBS https://www.idbs.com/en/ (acessado em 19/01)
	\item [45] Google www.google.com.br
    \item [46] Amazon www.amazon.com
    \item [47] Microsoft www.microsoft.com
    \item [48] Google Compute Engine https://cloud.google.com/compute/
    \item [49] GoGrid https://www.datapipe.com/gogrid/
    \item [50] CloudForge http://www.cloudforge.com
    \item [51] AppScale http://www.appscale.com
    \item [52] Abiquo http://www.abiquo.com
    \item [53] SalesForce http://www.salesforce.com/br/
    \item [54] Zendesk https://www.zendesk.com.br    
	\item [55] Breno Rodrigues Moura and Deric Lima Bacelar. Política para armazenamento de arquivos no zoonimbus, 2013. Monografia de graduação, Departamento de Ciência de Computação, Universidade de Brasília. 20, 21, 25
	\item [56] Gabriel Silva Souza de Oliveira. Acosched: um escalonador para o ambiente de nuvem federada zoonimbus, 2013. Monografia de graduação, Departamento de Ciência de Computação, Universidade de Brasília. 20, 21, 24
	\item [57] Ion Stoica, Robert Morris, David Liben-Nowell, David Karger, Frans Kaashoek, Frank Dabek, and Hari Balakrishnan. Chord: a scalable peer-to-peer lookup protocol for internet applications. Networking, IEEE/ACM Transactions on, 11(1):17–32, 2003. 20
	\item [58] Raj Srinivasan. RPC: Remote Procedure Call protocol Specification version 2 , Request For Coments (RFC 1831). http://tools.ietf.org/pdf/rfc1831.pdf, 1995. Acessado online em 27 de janeiro de 2016.
	\item [59] The Apache Software Fundation. Apache zookeeper. http://zookeeper.apache.org/, 2015. Acessado online em 15 de janeiro de 2016. vii, 21, 22
	\item [60] The Apache Software Fundation. Apache fundation. http://apache.org/, 2015. Acessado online em 15 de janeiro de 2016. 21
	\item [61] The Apache Software Fundation. Apache Avro. http://avro.apache.org/, 2012. Acessado online em 15 de janeiro de 2016
	\item [62] Douglas Crockford. The application/json Media Type for Javascript Object Notation (JSON) , Request For Coments (RFC 4627). http://tools.ietf.org/pdf/rfc4627.pdf, 1995. Acessado online em 05 de junho de 2015. 21
	\item [63] Roy Fielding, Jim Gettys, Jeffrey Mogul, Henrik Frystyk, Larry Masinter, Paul Leach, and Tim Berners-Lee. Hypertext Transfer Protocol–http/1.1 , Request For Coments (RFC 2616). http://tools.ietf.org/pdf/rfc2616.pdf, 1999. Acessado online em 05 de junho de 2015. 21
	\item [64] The Apache Software Fundation. Apache avro documentation. http://avro.apache.org/docs/1.7.7/, 2015. Acessado online em 05 de junho de 2015. 21
    \item [65] Thrift https://thrift.apache.org
    \item [66] Protocol Buffer https://developers.google.com/protocol-buffers/
\end{itemize}

%    \apendice{Apendice_Fichamento}{Fichamento de Artigo Científico}%    
%    \anexo{Anexo1}{Documentação Original \unbcic\ (parcial)}%
\end{document}%